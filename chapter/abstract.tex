\chapter*{Abstract}
The revolution of Massive Online Open Courses (MOOCs) brings great opportunities for millions of learners worldwide. Meanwhile, student-generated data on MOOCs could be effectively used to improve both the teaching and learning effectiveness.
In this study, considering the habit of replay the video, a data-driven learning interest recommendation system called Videomark is proposed to identify the specific concepts that might interest leaners through integrating both the learner’s logs and video subtitles in the Chinese-speaking environments.
Videomark provide a “keywords cloud” learning interest/difficult reminding system based on learners’ video watching logs and subtitles is proposed for promoting self-paced MOOC learning.
By identifying the hot video segments (via video seek event counts) and weighting the keywords of hot video segments, we are able to establish the “keywords cloud” of each learning topic.
This feature is valuable for learners to quick identify the most important or difficult concepts of each topic.
This is also useful for the teacher to more understand which parts of the contents of each topic are most difficult for the learners which can be further improved.
We hope that the proposed system help learners to review, consolidate, and clarify specific concepts in the video-based learning environment.
The overall idea and steps will be presented in this study

% {\let\thefootnote\relax\footnotetext{\ThankOne}}

\noindent \textbf{Keywords -}{MOOCs, self-paced learning, keywords cloud, subtitles.}
