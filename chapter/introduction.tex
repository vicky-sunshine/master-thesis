\chapter{Introduction}
Massive Online Open Courses (MOOCs) refer to an open educational resources,
which allows learners worldwide to take well-designed online courses of interest free of charge.
On MOOCs, learners watch the high-quality instructional videos made by professors from prestigious universities,
share their ideas and reflections on the discussion forum,
and use the online exercise system to evaluate their learning outcome.
Due to the fact that the MOOCs provide with high-quality self-directed learning environment without costing much for online learners,
MOOCs have been thought of as a contemporary way of 21-century learning.

There are two type of MOOCs, cMOOCs (connectionism MOOCs) and xMOOCs (instructionism MOOCs).
These two types of MOOCs are base on different philosophical positions underpinning,
cMOOCs focuses on connections between participants in particular on strong content contributions from the participants themselves \cite{yeager2013cmoocs},
xMOOCs, by contrast focus on instructor's design of the course.
Many famous MOOCs platform such as Coursera\cite{coursera}, edX\cite{edx}, and Udacity\cite{udacity} are belong to xMOOCs.

For current xMOOCs, the instructional videos play a significant role in the online learning process \cite{breslow:2013,seaton2014}.
In essence, the learning focuses in the form of visual and audial presented in the instructional videos.
Traditionally, video-based learning follows structured instructor-designed sequences for the better results.
Owing to the technological nature of the online stream video, it is found that many students drag the play bar replaying specific concept in the video for consolidating their understanding.
Therefore, many studies aim to improve the video-based learning environment by adding additional features in video-watching, such as embedded assessment, caption tool, as so on.

In view of the rapid development of data sciences,
more and more studies on educational data mining and learning analytics take the advantages of the learners’ data to optimize learning process.
For example, \cite{kim2014} develops a step-by-step annotations feature to improve the learning experience of existing how-to videos.
Study \cite{agrawal2015} constructs a system that recommends students videos best on their forum post, making a self-solved confusion system and meanwhile reducing the teaching load.
Therefore, considering the learning needs and the authentic learner data,
this study develops a data-driven learning interest recommendation system to promote self-paced learning by integrating educational data mining and word segmentation in the Chinese-speaking MOOC environment.
Videomark combines both the learning seek event counts and the subtitles of each video to automatically generate learning concepts for learners in friendly user interface.
Through the huge amount of video watching/seeking log data, the Videomark helps learners to quickly identify popular video seek events for consolidating their concept of the learning focus in hope of promoting better self-paced video-based learning environment.

The remainder of this thesis is organized as follows.
In Section~\ref{cha: Brief Review of Reinforcement Learning and Least-Square Policy Iteration}, we first introduce the Markov decision process and brief review the reinforcement learning algorithm which is called least-square policy iteration.
In Section~\ref{cha: System Model and Problem Formulation}, the energy management problem at the consumer side is examined. We designed our system model by considering a EMS center which want to regulate energy flow such as day-ahead energy purchasing, real-time energy purchasing, and energy dumping to minimize marginal cost and prolong battery life.
In Section~\ref{cha: Learning-Based Energy Management Policy with DoD Considerations}, the reinforcement learning based energy management problem is examined.
In Section~\ref{cha: Simulation}, the performance of purposed algorithm is examined.
