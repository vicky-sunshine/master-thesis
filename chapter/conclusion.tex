\chapter{Conclusion and Future Work}\label{cha:Conclusion}
In this thesis, a “keywords cloud” learning interest/difficult reminding system (LIDRS) based on learners’ video watching logs and subtitles has been developed for promoting self-paced MOOC learning.
By identifying the hot video segments (via video seek event counts) and weighting the keywords of hot video segments, we are able to establish the “keywords cloud” of each learning topic.
This feature is valuable for learners to quick identify the most important or difficult concepts of each topic.
This is also useful for the teacher to more understand which parts of the contents can be further improved for each topic.
To the best of our knowledge, this is the first content based interest analysis and recommendation system supports the Chinese-speaking and Chinese-subtitles environment.
Furthermore, our experiments of two course proves the proposed system improves users' learning performance and their learning engagement, both questionnaire and cloud usage record can match with each other to explain the same result.

Beyond design and implementation result,there is still a lot of improvement we could make to promote the usage or optimize the user perception in the future.
Base on our observation, keyword cloud is the feature user like the most but rare of them used Videomark to watch video non-sequentially, the reason of this phenomenon may cause by the user interface design or user's preference.
It still lacks of empirical evidence to explain the low Videomark usage rate.

\begin{description}
  \item[1.] Manually input subtitle
\end{description}

The proposed system analyze keyword in course is heavily rely on lecture video subtitle, our subtitle is manually generated firstly before the course begin and double checked.
However not every course has subtitle file for all the video, and regenerate subtitle file for all these course is not a efficient solution.
As a result, tools can auto generate Chinese speaking lecture is needs to be developed to break through the limitation.

\begin{description}
  \item[2.] User Interface
\end{description}

Base on the feedback of questionnaire the web page of Videomark will sometimes confuse user.
Our design to hide urls of keyword at the beginning of user access to this page will sometimes make user doesn't aware that the url section exist and this may be the reason why our usage rate of keyword url click is so low.
We should put more effort to find and use our features easily.

\begin{description}
  \item[3.] Other Category of Activity Record
\end{description}

In the proposed system, we analyze only seek event to find out the learning interest in the video, but actually our system also collected click stream data including play, pause and change play rate.
Combining these action, the analysis result may be more accurate.

Currently the “keywords cloud” is constructed for the Chinese-speaking and Chinese-subtitles environment.
More specific, innovative, and learner-centered features especially for the video-based learning environment can be further developed in the future.
