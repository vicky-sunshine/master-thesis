\chapter{Conclusion and Future Work}\label{cha:Conclusion}
In this thesis, a “keywords cloud” learning interest/difficult reminding system (LIDRS) based on learners’ video watching logs and subtitles has been developed for promoting self-paced MOOC learning.
By identifying the hot video segments (via video seek event counts) and weighting the keywords of hot video segments, we are able to establish the “keywords cloud” of each learning topic.
This feature is valuable for learners to quick identify the most important or difficult concepts of each topic.
This is also useful for the teacher to more understand which parts of the contents can be further improved for each topic.
Currently the “keywords cloud” is constructed for the Chinese-speaking and Chinese-subtitles environment.
It still lacks of empirical evidence to confirm the proposed feature for other languages.
More specific, innovative, and learner-centered features especially for the video-based learning environment can be further developed in the future.
The user experiences, effectiveness, or feedback of using the keywords cloud also worth to be further studied and analyzed.
