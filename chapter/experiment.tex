\chapter{Experiment}\label{cha:Experiment}

\section{Overview}
Since we are curious about whether Videomerk helps user or not, Videomark had been applied in two courses on ShareCourse.
In this chapter we will validate the effects of Videomark by records of user activities and usage of Videomark.
We will focus on user learning performance and learning engagement.

\section{Introduction to Computer Networks (2016-spring)}
The first course we applied Videomark in Introduction to Computer Networks (2016-spring), we analyzed the learning performance and learning engagement by comparing users' activity records between users' learned with and learn without Videomark feature.
Table \ref{table:computernet2016} shows the basic information of the course.
\begin{table}[H]
\centering
\caption{Base Information of Introduction to Computer Network}
\label{table:computernet2016}
\resizebox*{12cm}{!}{
\begin{tabular}{|l|l|}
\hline
course duration & 2016/2/15 - 2016/5/18 \\ \hline
chapter count   & 9                     \\ \hline
video count     & 97                    \\ \hline
student number & 2577                  \\ \hline
exam time & 2                           \\ \hline
pass students   & 113                   \\ \hline
pass rate       & 4.28\%                \\ \hline
\end{tabular}}
\end{table}

\subsection{Learning Performance}
We classify user enrolled this course in two categories, user learn with and learn without the help of keyword cloud.
If a user did click the word in keyword cloud, we classify the user into the side that uses keyword cloud, and vice versa.
However, user that never applied any exam will not classified into both categories.
Table \ref{table:computerperformancecompare} compare the average score in two exam, and the final pass rate of the course.

\begin{table}[H]
\centering
\caption{Comparison User's Learning Performance}
\resizebox{\columnwidth}{!}{
\label{table:computerperformancecompare}
\begin{tabular}{|l|l|l|}
\hline
                       & Users learn with keyword cloud & User learn without keywrod cloud \\ \hline
persons                & 12                             & 2080                             \\ \hline
1st exam average score & 91.26                          & 76.55                            \\ \hline
2ed exam average score & 91.19                          & 81.69                            \\ \hline
pass rate              & 83.3\%                         & 51.04\%                          \\ \hline
\end{tabular}}
\end{table}


\subsection{Learning Engagement}
In addition to learning performance, we also analyzed the frequency of video watch, forum activity records to observe whether Videomark will boost the learning engagement.
As same as the way we classify user to compare learning performance, the result of learning engagement comparison is showed in Table \ref{table:computeractivitycompare}
\begin{table}[H]
\centering
\caption{Comparison User's Learning Engagement}
\label{table:computeractivitycompare}
\resizebox{\columnwidth}{!}{
\begin{tabular}{|l|l|l|}
\hline
                            & Users learn with keyword cloud & User learn without keywrod cloud \\ \hline
average watch time per week & 124                            & 28.56                            \\ \hline
average forum post          & 6.1                            & 2.51                             \\ \hline
average forum reply         & 13.25                          & 3.61                             \\ \hline
average forum like give     & 6.95                           & 3.13                             \\ \hline
\end{tabular}}
\end{table}


\section{Network Security (2016-spring)}
The second course we applied Videomark is Network Security (2016-Spring), the course information shows in Table \ref{table:networksec2016}.
We do the same data-driven analysis on users' learning performance and learning engagement.
In addition, we observed the video watching pattern after enter the recommended keyword video segment to understand whether the feature is useful for them.
At the end of the course, we made a questionnaire to get users' feedback about user interface usage experience and feeling after using the Videomark feature.

\begin{table}[H]
\centering
\caption{Base Information of Network Security}
\label{table:networksec2016}
\resizebox*{12cm}{!}{
\begin{tabular}{|l|l|}
\hline
course duration & 2016/4/15 - 2016/6/21 \\ \hline
chapter count   & 7                     \\ \hline
video count     & 51                    \\ \hline
sutudent number & 2841                   \\ \hline
pass students   & xxx                   \\ \hline
pass rate       & xx\%                  \\ \hline
\end{tabular}}
\end{table}

\subsection{Performance of Learning}
\subsection{Engagement of Learning}
\begin{table}[H]
\centering
\caption{Comparison User's Learning Engagement}
\label{table:securityactivitycompare}
\resizebox{\columnwidth}{!}{
\begin{tabular}{|l|l|l|}
\hline
                            & Users learn with keyword cloud & User learn without keywrod cloud \\ \hline
average watch time per week & 124                            & 28.56                            \\ \hline
average forum post          & 6.1                            & 2.51                             \\ \hline
average forum reply         & 13.25                          & 3.61                             \\ \hline
average forum like give     & 6.95                           & 3.13                             \\ \hline
\end{tabular}}
\end{table}
\subsection{Questionnaire}
Our questionnaire is designed base on three research question.
\begin{description}
  \item[1.] Is Videomark improves users' learning performance.
  \item[2.] Is Videomark improves users' learning engagement.
  \item[3.] Is Videomark user interface easy to use or easy to learn.
\end{description}

Table \ref{table:questionnairePerformance} show the questions and result about improving learning performance the numbers in table is the percentage of the answer.
The result shows that half of user believe Videomark will positively effect their performance.

\begin{CJK*}{UTF8}{bsmi}
\begin{table}[H]
\centering
\caption{Performance of Learning}
\label{table:questionnairePerformance}
\resizebox*{\columnwidth}{6cm}{
\begin{tabular}{lllllll}
   & 學習成效增進程度                & 非常同意 & 同意   & 不確定  & 不同意  & \begin{tabular}[c]{@{}l@{}}非常\\ 不同意\end{tabular} \\ \hline
1. & 關鍵字雲讓我更容易記住影片中老師講的內容      & 34.7 & 48.6 & 15.3 & 1.4  & 0     \\
2. & 我對每個章節的關鍵字有深刻的印象          & 27.8 & 56.9 & 0    & 0    & 0     \\
3. & 我更容易記住與關鍵字相關的內容           & 30.6 & 51.4 & 16.7 & 1.4  & 0     \\
4. & 關鍵字雲使我在課程開始前能瞭解課程大致內容     & 29.2 & 48.6 & 19.4 & 0    & 0     \\
5. & 關鍵字雲讓我更容易了解課程的重點          & 26.4 & 58.3 & 12.5 & 2.8  & 0     \\
6. & 關鍵字雲讓我更容易了解章節內容的架構        & 27.8 & 43.1 & 26.4 & 2.8  & 0     \\
7. & 關鍵字雲對於我幫助我更快的複習課程         & 22.2 & 51.4 & 25   & 1.4  & 0     \\
8. & 關鍵字雲讓我更快找到課程中遇到的問題點       & 16.7 & 43.1 & 29.2 & 11.1 & 0     \\
9. & 關鍵字雲使我在日常生活中網路安全的應用有更多的了解 & 25   & 43.1 & 26.4 & 5.6  & 0  \\  \hline
\end{tabular}}
\end{table}
\end{CJK*}

Table \ref{table:questionnaireEngagement} shows the feedback about learning engagement effects of Videomark.

\begin{CJK*}{UTF8}{bsmi}
\begin{table}[H]
\centering
\caption{Engagement of Learning}
\label{table:questionnaireEngagement}
\resizebox*{\columnwidth}{7cm}{
\begin{tabular}{lllllll}
         & 學習投入增進程度                         & 非常同意 & 同意   & 不確定  & 不同意  & \begin{tabular}[c]{@{}l@{}}非常\\ 不同意\end{tabular} \\ \hline
1.       & 關鍵字雲讓我更想解決我不懂的問題            & 11.1 & 62.5 & 19.4 & 6.9  & 0                                                \\
2.       & 關鍵字雲的功能讓我更有動力上ShareCourse學習 & 16.7 & 30.6 & 31.9 & 20.8 & 0                                                \\
3.       & 關鍵字雲讓我在課程的重點更加專注            & 18.1 & 54.2 & 19.4 & 8.3  & 0                                                \\
4.       & 關鍵字雲增加我在ShareCourse上課的次數    & 11.1 & 25   & 48.6 & 15.3 & 0                                                \\
5.       & 我會重複觀看關鍵字雲標註的內容片段           & 19.4 & 45.8 & 22.2 & 12.5 & 0                                                \\
6.       & 我會更常思考關鍵字雲裡面出現的重點           & 16.7 & 66.7 & 13.9 & 2.8  & 0                                                \\
7.       & 我會特別觀注關鍵字出現的討論串             & 23.6 & 44.4 & 26.4 & 5.6  & 0                                                \\
8.       & 關鍵字雲使我更想要在討論區發表問題           & 5.6  & 23.6 & 59.7 & 9.7  & 1.4                                              \\
9.       & 關鍵字雲讓我更積極回應討論區的問題           & 6.9  & 25   & 56.9 & 11.1 & 6.9                                              \\
10.      & 配合關鍵字雲學習讓我感到心情愉快            & 13.9 & 50   & 30.6 & 5.6  & 0                                                \\
11.      & 我會期待看到新一週的關鍵字雲              & 15.3 & 43.1 & 30.6 & 11.1 & 0                                                \\
12.      & 看到新一週的文字雲後我會更想觀看本週課程        & 18.1 & 44.4 & 33.3 & 4.2  & 0                                            \\ \hline
\end{tabular}}
\end{table}
\end{CJK*}

Table \ref{table:questionnaireEngagement} shows the feedback about user interface experience of Videomark. 

\begin{CJK*}{UTF8}{bsmi}
\begin{table}[H]
\centering
\caption{Usage of Keyword Cloud}
\label{table:questionnaireUsage}
\resizebox*{\columnwidth}{5cm}{
\begin{tabular}{lllllll}
   & 使用經驗                            & 非常同意 & 同意   & 不確定  & 不同意 & \begin{tabular}[c]{@{}l@{}}非常\\ 不同意\end{tabular} \\ \hline
1. & 我能夠容易地找到展開關鍵字雲的按鈕               & 22.2 & 52.4 & 16.7 & 6.9 & 0                                                \\
2. & 展開後的關鍵字雲的位置非常明顯                 & 25   & 51.4 & 15.3 & 6.9 & 1.4                                              \\
3. & 我能容易的找到在點擊關鍵字後出現的關鍵字連結          & 13.9 & 63.9 & 16.7 & 2.8 & 2.8                                              \\
4. & 我能清楚地看出關鍵字雲中字的大小清楚的分別出各個關鍵字的重要性 & 25   & 56.9 & 16.7 & 1.4 & 0                                                \\
5. & 關鍵字雲的影片連結能帶領我到正確的影片位置           & 13.9 & 63.9 & 19.4 & 2.8 & 0                                                \\
6. & 我認為關鍵字雲放置的地方非常合適                & 16.7 & 58.3 & 20.8 & 4.2 & 0                                                \\
7. & 關鍵字雲內的關鍵字是課程真正重要的概念             & 19.4 & 50   & 29.2 & 0   & 1.4                                            \\ \hline
\end{tabular}}
\end{table}
\end{CJK*}
